\documentclass[linenumbers,twocolumn]{aastex631}

% Packages
\usepackage[utf8]{inputenc}
\usepackage{graphicx}
\usepackage{amsmath}
\usepackage{amssymb}
\usepackage{enumitem}
\usepackage{ulem}
% \usepackage{hyperref}

% Editing commands
\newcommand{\mm}[1]{{\textcolor{purple}{\bf #1}}}

% Make upright subscripts and superscripts in Mathmode.
\def\subinrm#1{\sb{\mathrm{#1}}}
{\catcode`\_=13 \global\let_=\subinrm}
\mathcode`_="8000
\def\supinrm#1{\sp{\mathrm{#1}}}
{\catcode`\^=13 \global\let^=\supinrm}
\mathcode`^="8000
\def\upsubscripts{\catcode`\_=12 } \def\normalsubscripts{\catcode`\_=8 }
\def\upsupscripts{\catcode`\^=12 } \def\normalsupscripts{\catcode`\^=7 }

\newcommand{\vdag}{(v)^\dagger}
\newcommand\aastex{AAS\TeX}
\newcommand\latex{La\TeX}

% Title
\shorttitle{GRB Prompt Emission Simulation}
\shortauthors{Michael Moss}

\begin{document}

\upsubscripts
\title{GRB Prompt Emission Simulation}

\author{Michael Moss}
\affiliation{The Department of Physics, The George Washington University, 725 21st NW, Washington, DC 20052, USA}
\affiliation{Astrophysics Science Division, NASA Goddard Space Flight Center, Greenbelt, MD 20771, USA}
\affiliation{Sorbonne Universit\'e, CNRS, UMR 7095, Institut d'Astrophysique de Paris, 98 bis Arago, 75014 Paris, France}


\correspondingauthor{Michael Moss}
\email{mikejmoss3@gmail.com}


\begin{abstract}
Abs
\end{abstract}

\keywords{Gamma-Ray Bursts (629)}

\section{Introduction} \label{sec:intro}
\section{Jet Dynamics} \label{sec:jet dynamics}

Input parameters for the jet dynamics simulations are listed in Table \ref{tab: jet dynamic params}. In this section we provide context for the input parameters and derive some relevant parameter from the input parameters. Following this, we derive the calculated emission parameters (for example, the temperature of the black body). The parameter space of the code can be limited in two ways, either using parameters extracted from fits to real GRB spectra, which we call option (a), or by creating an library of spectra using reasonable values for each parameter, which we call option (b). If the value of the parameter cannot be constrained by observations, then option (a) and (b) are equivalent.

The duration of GRB prompt emission is not well defined. Some GRBs may last only a fraction of a second while others last hundreds of seconds. Additionally, the observed duration does not directly indicate the duration of the wind or central engine, it only provides a hard upper limit. From the observed duration of a GRB, e.g., T$_{90,obs}$, we are able to limit the possible duration of the wind, 

\begin{align}
	t_w = \frac{\text{T}_{90,obs}}{1+z} * f_w
\end{align}

where $z$ is the redshift of the source and $f_w$ is some fraction used to account for the fact that $t_w < $T$_{90,obs}$. We do know the exact value of $f_w$, but we can use a few values, e.g. 0.75, 0.85, 1). If we do not use observations to define the wind in our simulation, we can use various durations ranging from 0.1 sec to 10 sec.

The time resolution of our simulations will be limited by the ejection time between the shells.

The number of shells in the jet can defined by the total duration of the wind, $t_w$, and the ejection time between shells, $\Delta t_{ej}$, 

\begin{align}
	N = \frac{t_w}{\Delta t_{ej}}
\end{align}

The magnetization in the jet is an important parameter which dictates the acceleration and emission processes possible in the jet. If $\sigma=0$, the jet is in a purely fireball model, meaning the acceleration of the jet is purely thermal and that there can be no magnetic reconnection because there is no magnetic flux in the jet. If on the other hand, $\sigma \geq 1$, the magnetic fields in the jet are strong enough to suppress internal shocks and magnetic reconnection events are expected. In the case of a passive magnetic field that is carried by the outflow without contributing to its acceleration \citep{2001A&A...369..694S} the passive magnetization is given

\begin{align}
	\sigma_{passive} = \frac{1-\epsilon_{th}}{\epsilon_{th}}
\end{align} 

corresponding to a pure and complete thermal acceleration (and indicating no magnetic acceleration). If there is any magnetic acceleration $\sigma < \sigma_{passive}$.

Similar implications can be obtained from $\epsilon_{th}$, when $\epsilon_{th}$ we are in a pure fireball model and the upper limit can be defined as $\epsilon_{th}<\frac{1}{1+\sigma}$.

Using observations of GRB prompt emission we can define a value of the injected energy, $\dot{E}$. We can use the observed isotropic injected energy rate of the non-thermal component in the gamma-ray regime, $\dot{E}_{\gamma,NT,iso}$, to define the injected energy rate in the jet, 

\begin{align}
	\dot{E} = (1+\sigma)\frac{\dot{E}_{\gamma,NT,iso}}{f_{rad}\epsilon_e}
\end{align}

where $f_{rad}$ is the efficiency of the non-thermal acceleration process, e.g., for internal shocks $f_{rad}$ is determined by the Lorentz factor of two colliding shells, but is typically around a few percent. Instead of using the observed values for a particular value, we can simulate the wind using injection energies ranging from $10^{50}$ to $10^{54}$ erg. 

We can derive the various energies before acceleration

\begin{align}
	\dot{E}^{init}_{th} &= \epsilon_{th}\dot{E} \\
	\dot{E}^{init}_{mag} &= (1-\epsilon_{th})\dot{E} \\
	\dot{E}^{init}_{kin} &= 0
\end{align}

and energies after acceleration

\begin{align}
	\dot{E}^{final}_{th} &= 0 \\ 
	\dot{E}^{final}_{mag} &= \frac{\sigma}{1+\sigma}\dot{E} \\
	\dot{E}^{final}_{kin} &= \frac{\dot{E}}{1+\sigma}
\end{align}

where $\dot{E}_{th}$, $\dot{E}_{mag}$, $\dot{E}_{kin}$ are the amounts of energy in the form of thermal, magnetic, and kinetic, respectively. We note $\dot{E} = (1-\cos\theta_j)\dot{E}_{iso}$.


\subsection{Shell Distribution}

We begin by assigning the initial Lorentz factor, mass, emission time since jet launch, radius, and status of the N shells in the jet. 

The distribution of the shell Lorentz factors can be specified by the user before running the jet dynamics simulation. Currently, there is a step function and a oscillatory function implemented. For the step function, two Lorentz factors must be specified to specify the upper $\Gamma_1$ and low steps $\Gamma_2$. The the fraction of the total mass with $\Gamma_1$ must also be specified, $f=M_1/M_{tot}$

\begin{align}
	N(\Gamma_1) &= \frac{N_{tot}}{\frac{(1-f)*\Gamma_2}{f*\Gamma_1} + 1} \\
	N(\Gamma_2) &= N_{tot} - N(\Gamma_1) \\ 
\end{align}

 For the oscillatory distribution, we adapt the function from Equation 16 in \citet{2013A&A...551A.124H}, but has been made more general,

 \begin{align}
	\Gamma_i = \Gamma_{mean} * \left(1+A\cos\left(\pi f\left(1-\frac{i}{N_{tot}}\right)\right)\right)*e^{-\lambda \frac{i}{N_{tot}}} 
 \end{align}

where $\Gamma_{mean}$ is the mean Lorentz, A is the amplitude, f is the frequency, i is the shell number, and $\lambda$ is the decay constant.

The mass of each shell is defined as

\begin{align}
	M_i &= \frac{\bar{\Gamma}}{\Gamma_i} \\ 
\end{align}

where $\bar{\Gamma}$ is the mean Lorentz factor. Currently, this is a dimensionless quantity, later it is multiplied by the $\bar{M}$ to give dimensions, where

\begin{align}
\bar{M} = \frac{\dot{E}_{iso} \Delta t_e}{\bar{\Gamma}c^2} \text{ g}
\end{align}

The emission time of each shell can either be supplied as a list by the user or will be calculated if only $\Delta t_e$ is given. The emission time is the $t_e = -\Delta t_e * i$, where i is the shell number. Notice the emission time is negative, this doesn't have a physical reason, its just how I coded it. 

The radius of the jet computed as,

\begin{align}
	R_{0,i} = \beta_i * t_{e,i}\text{ cm/c}
\end{align}

where $\beta_i = \sqrt(1 - \Gamma^{-2})$ Notice that these are in units of $R/c$, for the remainder of this discussion distances will be in units of light seconds. Distances should be multiplied by $c$ in order to obtain distance units. Recall that the values of $t_{e,i}$ are negative, so these radii will also be negative. We set $R = 0 = R_{0,0}$ and all shells begin at negative distances.

The status of a shell indicates whether it is currently active or not. An active shell is one that is currently propagating through the jet. An inactive shell is one that has not been launched yet or alternatively has already collided. 


% \begin{deluxetable*}{llll}
% \tablecaption{ List of parameters used in the jet dynamic simulations, their variable names and default values in the code, as well as a description of each. \label{tab: jet dynamic params}}
% \tablewidth{0pt}
% \tablehead{
% \colhead{Parameter symbol} & \colhead{Variable Name (in code)} & \colhead{Default Value} &  \colhead{Description}
% }
% \decimalcolnumbers
% \startdata
% N & numshells & 5000 & Number of shells in the jet. \\
% $\Delta$t$_e$ & dte & 0.002 s & Time between shell launching. \\
% $\alpha_e$ & alpha$\_$e & 1/3 & Fraction of energy which goes into the electrons. \\
% $\alpha_m$ & alpha$\_$m & 1 & Between 0.1 - 1, fraction of the dissipated energy \\
% & & & which goes into magnetic fluctuations.\\
% $\alpha_b$ & alpha$\_$b & 1/3 & Fraction of energy which goes into the magnetic field. \\
% $\zeta$ & zeta & 0.001 & Fraction of electron which are accelerated. \\
% $\mu$ & mu & 1.74 & Index of the fluctuation spectrum. \\
% $\dot{E}_{iso}$ & E$\_$dot$\_$iso & $10^{53}$ erg/s & Isotropic energy injection rate. \\
% t$_b$ & tb & 0 s & Break out time. \\
% $\theta$ & theta & 0.1 rad & Opening angle of the jet. \\
% r$_{open}$ & r$\_$open & $10^6$ cm & Opening radius of the jet. \\
% r$_{sat}$ & r$\_$sat & $10^8$ cm & Saturation radius. \\
% $\epsilon_{th}$ & eps$\_$th & 0.03 & Fraction of injected energy turned in thermal. \\
% $\sigma$ & sigma & 0.1 & The magnetization of the outflow. \\
% Lorentz Distribution & LorentzDist & ``step'' & Describes which Lorentz distribution \\
% & & & to use for the shells. \\
% Lorentz Dist. Param File & ShellDistParamsFile & ``Default'' & File that contains the parameters to create the\\
% & & & distribution of jet shells. \\
% entry1 & entry2 & entry3 & entry4 \\
% \enddata
% \end{deluxetable*}

\begin{deluxetable*}{llll}
\tablecaption{List of input parameters used in the jet dynamic simulations, the typical values or expression and a description of each is provided. The parameter space of the code can be limited in two ways, either using parameters extracted from fits to real GRB spectra, which we call option (a), or by creating an library of spectra using reasonable values for each parameter, which we call option (b). If the value of the parameter cannot be constrained by observations, then option (a) and (b) are equivalent.\label{tab: jet dynamic params}}
\tablewidth{0pt}
\tablehead{
\colhead{Parameter symbol} & \colhead{Typical Values} &  &  \colhead{Description} \\ 
& (a) & (b) &
}
\startdata
$t_w$ & $\frac{\text{obs duration}}{1+z}*(0.75, 0.85, 1)$ sec & (0.1, 0.3, 1, 3, 10) sec & Duration of the wind. \\
$\Delta t_{ej}$ & ($10^{-2}$, $10^{-3}$) sec & & Time Resolution of the simulation, or \\
& & & time between the ejection of each shell. \\
$\Gamma_{min}$ & (50, 100, 200) & & Minimum Lorentz factor for any shell. \\
$\Gamma_{max}$ & (2, 4, 6)*$\Gamma_{min}$ & & Maximum Lorentz factor for any shell. \\
$f_{\Gamma}$ & (0.2, 0.4, 0.6, 0.8) & & Fraction of shells which have $\Gamma_{max}$. \\
$\sigma$ & (0, $10^{-2}$, $10^{-1}$) & & Magnetization of the jet. \\
$\epsilon_e$ & (0.1, 1/3) &  & Fraction of energy stored in electrons. \\
$\epsilon_B$ & ($10^{-4}$, $10^{-3}$, $10^{-2}$, $10^{-1}$) & & Fraction of energy stored in the magnetic field. \\
$\epsilon_{Th}$ & ($10^{-3}$, $10^{-2}$, $10^{-1}$, 1) & & Fraction of energy stored as Thermal energy. \\
$\dot{E}$ & $(1+\sigma)*\frac{\dot{E}_{\gamma,NT,iso}}{f_{rad}\epsilon_e}$ erg/s & ($10^{50}$,$10^{51}$,$10^{52}$,$10^{53}$,$10^{54}$) erg/s & Injected energy rate. \\
$\zeta$ & ($10^{-3}$, 3*$10^{-3}$, $10^{-2}$, 3*$10^{-2}$) & & Fraction of electrons which are accelerated. \\
$p$ & (2.2, 2.5, 2.8, 3.4) &  & Power law index of electron distribution. \\
$\theta_j$ & 0.1 rad &  & Opening angle of the jet. \\
$l$ & $10^{6}$ cm & & Opening radius of the jet. \\
\enddata
\end{deluxetable*}


\subsection{Emission Output Parameters}
\subsubsection{Thermal Shell Dynamics and Emission}

Once we have assigned the initial Radius, Lorentz factor, Mass, and emission time of each shell, we can begin to calculate the dynamics of the shells. We calculate when the shells will cross their respective photospheres and we calculate when/where collisions between the shells will occur. 

We have found that all the shells pass their respective photospheres much before the first collision between two shells (in the source frame).

The photospheric radius and crossing time of each shell is given by:

\begin{align}
	R_{phot,i} &= \frac{0.2 \dot{E}_{iso}}{(1+\sigma)8\pi c^4 \Gamma_i^3} \text{ cm/c}\\
	t_{phot,i} &= \frac{R_{phot} - R_{0,i}}{\beta_i}\text{ s}
\end{align}

where $R_{0,i}$ is the initial radius of the $i$th shell.

For all thermal emission events, the arrival time of the photons and duration of the emission is given by

\begin{align}
	t_a &= (t_e - R_{phot}) \text{s}\\
	\Delta t_{obs} &= \frac{R_{phot}}{2\Gamma_r^2} (1+z) \text{ s}
\end{align}

Following the prescription from \citet{2013A&A...551A.124H}.

\begin{align}
	\Phi &= \theta^{-2/3} (R_{phot}*c)^{-2/3} (R_{open})^{2/3} \Gamma^{2/3}\\
	&\approx \dot{E}_{iso}^{-2/3} \Gamma^{8/3} \\
	T0 &= \left(\frac{\dot{E}_{th}^2 \theta^2}{4\pi a c R_{open}^2} \right)^{1/4} \text{ K} \\ 
	T_{phot,obs} &= T0*\Phi*(1+z) \text{ K} \\
	L_{phot} &= \frac{\theta^2 \dot{E}_{th} \Phi }{4} \text{ erg/s}
\end{align}

where $a$ is the radiation constant. 

Output parameters calculated from the model for thermal emission in Table \ref{tab: thermal data output}

\begin{deluxetable*}{ll}
\tablecaption{Output parameters calculated from the model for thermal emission \label{tab: thermal data output}}
\tablewidth{0pt}
\tablehead{
\colhead{Parameter} & \colhead{Description}
}
\decimalcolnumbers
\startdata
$t_{e,i}$ & The time when shell $i$ crosses the photosphere. Also will be the emission time. \\
$t_{a,i}$ & The arrival time of photons produced by shell $i$. \\
$\Delta t_{i}$ & Duration width of the emission of shell $i$. \\
$T_{i}$ & Temperature of shell $i$ at emission. \\
$L_{i}$ & Emitted luminosity of shell $i$. \\
$R_{phot,i}$ & Radius of the photosphere for shell $i$. \\
\enddata
\end{deluxetable*}


\subsubsection{Internal Shock Shell Dynamics and Emission}

When considering the collisions between shells, we take each pair of adjacent active shells (shells which have already been launched) and calculate the time till they collide (see Appendix A). Selecting the minimum time until collision, $t_{coll,min}$, we move all shells according to $R_{new,i} = R_{old,i} + ()\beta_i * t_{coll,min})$. The two shells that collide will have an initial average Lorentz factor given by

\begin{align}
	\Gamma_r \sim \sqrt(\Gamma_s \Gamma_r)
\end{align}

for the slow and rapid shells. Once the momentum and energy of the shells has been fully redistributed the final Lorentz factor will be 

\begin{align}
	\Gamma_f = \sqrt(\Gamma_s \Gamma_r \frac{m_1 \Gamma_1 + m_2 \Gamma_2}{m_1\Gamma_2 + m_2\Gamma_1})
\end{align}

The new mass will be the sum of the two colliding shell masses. 

For all collision events, the arrival time of the photons and duration of the emission is given by

\begin{align}
	t_a &= (t_e - R_{coll})\\
	\Delta t_{obs} &= \frac{R_{coll}}{2\Gamma_r^2} (1+z)
\end{align}

To calculate the emission from the internal shocks, we follow the procedure detailed in \citet{1998MNRAS.296..275D}. Assuming synchrotron emission from electrons accelerated in a magnetic field. 

The average energy dissipated per proton in a shock is given by

\begin{align}
	\epsilon &= (\Gamma_{int} - 1) m_p c^2 \\
	& \text{ where } \Gamma_{int} = \frac{1}{2}\left[\left(\frac{\Gamma_1}{\Gamma_2}\right)^{1/2} + \left(\frac{\Gamma_2}{\Gamma_1}\right)^{1/2}\right]
\end{align}

where $\Gamma_{int}$ is the Lorentz factor for internal motions in the shocked material.

The comoving proton number density is given by 

\begin{align}
	n \approx \frac{\dot{M}}{4\pi r^2 \bar{\Gamma}m_p c} \approx \frac{\dot{E}_{kin}}{4\pi r^2 \bar{\Gamma}^2m_p c^3} \text{ 1/cm}^3
\end{align}

which allows us to calculate the equipartition magnetic field 

\begin{align}
	B_{eq} \approx (8\pi \epsilon_B n \epsilon)^{1/2} \text{ (erg/cm$^3$)}^{1/2}
\end{align}

where $\epsilon_B\lesssim1$. 

The typical synchrotron energy of the emitting electrons is given by

\begin{align}
	E_{syn} = 50 \frac{\Gamma_r}{300}\frac{B_{eq}}{1000\text{G}}\left(\frac{\Gamma_e}{100}\right)^2 \text{ eV}
\end{align}

This can be placed in the comoving frame 

\begin{align}
	E_{syn}^0 = E_{syn}/\Gamma_r \text{ eV}
\end{align} 

For electrons with a large enough Lorentz factor to produce gamma-rays via synchrotron emission, $\Gamma_e$ can be expressed using \citet{1996ApJ...461L..37B} who considered the scattering of electrons by turbulent magnetic field fluctuations and found 

\begin{align}
	\Gamma_e \sim \left[ \left(\frac{\alpha_M}{\xi}\right)\left(\frac{\epsilon}{m_ec^2}\right) \right]^{-1/(3-\mu)}
\end{align}

where $\alpha_M$ is the fraction fo the dissipated energy which goes into magnetic field fluctuations, $\xi$ is the fraction of electrons which are accelerated, and $\mu$ is the index of the fluctuation spectrum. For simplicity, we have currently used $\Gamma_e=10^4$.

Although we do not calculate the Inverse Compton (IC) emission in this work, we still must calculate the amount of electrons scattered to higher energies due to Compton scatter. This also requires considering when the particles are in the Klein-Nishina regime. Using the Klein-Nishina cross section occurs when $w>>1$, where

\begin{align}
	w = \frac{\Gamma_e E_{syn}^0}{m_e c^2} \approx 33 \frac{B_{eq}}{1000\text{G}}\left(\frac{\Gamma_e}{10^4}\right)^3
\end{align}

In the Klein-Nishina regime, the fraction of electrons which are scattered up to higher energies is given by 

\begin{align}
	\alpha_{IC} &= \frac{Q_{IC}}{(1+Q_{IC})} \text{ for } w < 1 \\
	\alpha_{IC} &= \frac{Q_{IC}/w}{(1+Q_{IC}/w)} \text{ for } w >> 1 
\end{align}

where $Q_{IC}$ is the Compton parameter and can be defined as 

\begin{align}
	Q_{IC} = \tau_{*}\Gamma_e^2
\end{align}

(Note: commonly in the literature, the Compton is labeled with a $Y$) where $\tau_{*}$ is the optical depth of the shell with mass $M_{*}$ and radius $r_{*}$ which contains relativistic electrons, 

\begin{align} \label{eq: QIC p1}
	\tau_{*} &= \frac{\kappa_T M_{*}}{4 \pi r_{*}^2}
\end{align}

where $\kappa_t$ is the Thomson opacity. The mass $M_{*}$ can be estimated

\begin{align} \label{eq: QIC p2}
	M_{*} &= \frac{t_{syn}}{1+Q_{IC}} \dot{M}_shock \\
	&\text{where}\\
	t_{syn} &= 6*\left(\frac{\Gamma_e}{100}\right)^{-1} \left(\frac{B_{eq}}{1000\text{G}}\right)^{-2}
\end{align}

is the synchrotron time of the relativistic electrons and $\dot{M}_{shock}$ is the mass flow rate across the shock, both in the comoving frame of the shocked material. Since the shock moves with a Lorentz factor $\sim \bar{\Gamma}$, $\dot{M}_{shock}$ can be approximated by 

\begin{align} \label{eq: QIC p3}
	\dot{M}_{shock} &\approx \frac{\dot{M}}{\bar{\Gamma}} \\
	&= \frac{\dot{E}_{iso}}{c^2 \bar{\Gamma}^2} 
\end{align}

From Equations \ref{eq: QIC p1}, \ref{eq: QIC p1}, and \ref{eq: QIC p1}, we can form the relation

\begin{align}
	Q_{IC} &= \frac{\kappa_T \dot{M}_{shock} t_{syn} \Gamma_e^2}{4\pi r_{*}^2 (1+Q_{IC})}\\ 
	Q_{IC}(1+Q_{IC}) &= \frac{\kappa_T}{4 \pi r_{*}^2} \frac{\dot{E}_{iso}}{c^2 \bar{\Gamma}^2} \Gamma_e^2 t_{syn}\\
	Q_{IC}(1+Q_{IC}) &= 150 \frac{\kappa_T}{\pi r_{*}^2} \frac{\dot{E}_{iso}\Gamma_e}{c^2 \bar{\Gamma}^2} \left(\frac{B_{eq}}{1000\text{G}}\right)^{-2}
\end{align}

The fraction of electrons which remain to emit synchrotron emission is simply 

\begin{align}
	\alpha_{syn} = 1 - \alpha_{IC}
\end{align}

Lastly, the energy dissipated in a collision between two shells is given by

\begin{align}
	e = min(M_1,M_2) c^2 (\Gamma_1 + \Gamma_2 - 2*\Gamma_r) * \epsilon_e * \alpha_{syn} \text{ erg}
\end{align}

Where we use the smaller mass of the two colliding shells.

Although, this is the emission that occurs from two collisions, two more constraints must be met in order to be observed: (i) the relative velocity between the two shells must be larger than the local sound speed and (ii) the wind must be transparent to the emitted photons. The relative velocity between two layers is given by 

\begin{align}
	\frac{v_{rel}}{c} \approx \frac{\Gamma_1^2 - \Gamma_2^2}{\Gamma_1^2 + \Gamma_2^2}
\end{align}

where we have adopted a sound speed of $v_s/c = 0.1$. \citet{1998MNRAS.296..275D} have checked that other choices make little difference in the results since the main contribution to the burst comes from shocks with large differences in the Lorentz factors, e.g., $\frac{\Gamma_r}{\Gamma_s} \gtrsim 2$. 

The transparency of the wind to the emitted photons is 

\begin{align}
	\tau = \kappa_T \sum_{i>i_{shock}} \frac{m_i}{4\pi r_i^2}
\end{align}

where $m_i$ and $r_i$ are the mass and radius of shell $i$. The sum over indices $i$ larger than $i_{shock}$, which corresponds to the colliding shells. 

Output parameters calculated from the model for synchrotron emission in Table \ref{tab: synchrotron data output}

\begin{deluxetable*}{ll}
\tablecaption{Output parameters calculated from the model for synchrotron emission \label{tab: synchrotron data output}}
\tablewidth{0pt}
\tablehead{
\colhead{Parameter} & \colhead{Description}
}
\decimalcolnumbers
\startdata
$t_{e,i}$ & Time of the $i$th collision (source frame). \\
$t_{a,i}$ & The arrival time of photons produced the $i$th collision. \\
$\alpha_{syn,i}$ & Fraction of energy which went into synchrotron elections for the $i$th collision. \\
$B_{eq,i}$ & Magnetic field in equipartition for the $i$th collision. \\
$\Gamma_{e,i}$ & Average Lorentz factor of the electrons in the $i$th collision.\\
$E_{syn}$ & Typical synchrotron energy in the $i$th collision. \\
$\Gamma_{r,i}$ & =$\sqrt{\Gamma_s\Gamma_r}$, initial averaged Lorentz factor of the $i$th collision. \\
$e_{diss,i}$ & Energy dissipated in the $i$th collision. \\
$\Delta t_{i}$ & Duration width of the emission of shell $i$. \\
$\tau_i$ & Optical depth at the $i$th collision. \\
$v_{rel,i}$ & Relative collision speed between the slow and fast shells of the $i$th collision. \\
\enddata
\end{deluxetable*}


\section{Spectrum Generation} \label{sec:spectrum}
\subsection{Thermal Spectrum}

\subsection{Synchrotron Spectrum}

Similar to how the thermal spectrum is created, the non-thermal spectrum generated from the GRB prompt emission simulations for a particular time-interval is created from the summation of spectra created by each emission event which occurred within time-interval.

The spectrum used for each emission event is therefor an important consideration. A simple first step is to assume a Broken Power Law (BPL) where the low and high energy power law indices are given by $\alpha=-1$ and $\beta\leq-2.5$. The value of alpha is an active discussion within the field and ranges between -0.6 to -1.5. The peak energy of the distribution can be determined by synchrotron energy calculated in the simulation, $E_{synch}$.

Alternatively, if we assume that synchrotron emission is produced by internal shocks events in the jet, we can use a standard definition of the synchrotron spectrum \citep{1998ApJ...497L..17S}. The variable parameters in this definition are the minimum Lorentz factor of the electron population and the cyclotron Lorentz factor, $\Gamma_m$ and $\Gamma_c$, respectively. These can be defined in terms of the parameters used in our simulations.

\begin{align}
	\Gamma_m &= \frac{p-2}{p-1} \frac{\epsilon_e}{\zeta} \frac{m_p}{m_e} \left(\frac{\epsilon}{c^2}\right) \\
	\Gamma_c &= \frac{const}{B^2 t'_{dyn}} \Rightarrow \frac{R}{\Gamma c}
\end{align}

where $\epsilon/c^2 = (\Gamma_{int} -1)$. These Lorentz factors can be directly turned to frequencies, $\nu_m$ and $\nu_c$, respectively. 


\section{Light Curve Generation} \label{sec:light curve}

The light curve of a simulated GRB is generated by taking the sum of all counts in the spectra created for each time bin of the light curve. 


% \begin{figure}[!ht]
%     \centering
%     \includegraphics[width=0.65\textwidth]{figures/example.png}
%     \caption{}
%     \label{fig: example fig}
% \end{figure}

% \begin{deluxetable*}{ll}
% \tablenum{3}
% \tablecaption{ \label{tab: example table}}
% \tablewidth{0pt}
% \tablehead{
% \colhead{Col1} & \colhead{Col2}
% }
% \decimalcolnumbers
% \startdata
% entry1 & entry2 \\
% \enddata
% \end{deluxetable*}



\newpage
\bibliographystyle{aasjournal}
\bibliography{bibliograpy-list}


\begin{appendix}

\section{Shell Collision Time and Radius}

Consider two shells traveling along the same axis. We will label the radius, Lorentz factor,  more rapid shell with $r$ and the slower shell with $s$

Consider two shells launched into from a central engine into a collimated relativistic jet. The second shell is launched with a delay of $\Delta t_e$. We can describe the position of the shells at a time $t$ with

\begin{align}
	R_1(t) &= R_{1,0} + c\beta_1 t \\
	R_2(t) &= R_{2,0} + c\beta_2 t
\end{align}

where $R_{i,0}$ is the initial radius of the launched shell and $\beta_i = v_i/c$. Since we are only considering these two shells, we can set the zero position at $R_{2,0}$, 

\begin{align}
	R_1(t) &= L + c\beta_1 t \\
	R_2(t) &= c\beta_2 t
\end{align}

where $L$ is the distance between the two shells at the launch of the second shell

\begin{align}
	L &= c\beta_1 \Delta t_e\\ 
	&\rightarrow R_1(t) = c\beta_1(t+\Delta t_e)
\end{align}

Alternatively, we could have set the zero to $R_{1,0}$ and found 

\begin{align}
	R_1(t) &= c\beta_1 t \\
	R_2(t) &= c\beta_2 (t-\Delta t_e)
\end{align}

The case where the second shell launched is slower than the first is trivial and results in no collision between the two. In the case where the second shell is more rapid than the first, there will eventually be a collision between the two shells. The time until collision can be written as

\begin{align}
	R_1(t_{coll}) &= R_2(t_{coll}) \\ 
	L + c\beta_1t_{coll}&= c\beta_2t_{coll}\\
	\rightarrow t_{coll}&= \frac{L}{c(\beta_2 - \beta_1)}\\
	t_{coll}&= \frac{c\beta_1 \Delta t_e}{c(\beta_2 - \beta_1)}\\
	t_{coll}&= \frac{\beta_1 \Delta t_e}{(\beta_2 - \beta_1)}
\end{align}

Substituting this back in the expression for the position we can find the collision radius, 

\begin{align}
	R_1(t_{coll}) &= L + c\beta_1 t_{coll} \\
	&= L+c\beta_1\left(\frac{L}{c(\beta_2-\beta_1)}\right) \\
	&= L\left(1+\frac{\beta_1}{(\beta_2-\beta_1)}\right)\\
	R_{coll} &= L\left(\frac{\beta_2}{\beta_2-\beta_1}\right) \label{eq: rcoll}
\end{align}

These calculations can be done for more than two shells, but keep in mind when calculating the time and position of the collisions after the initial collision the distance between them (i.e., L) will just be the difference in their positions (not their ``initial'' separation, but the separation at the time of calculation).

When calculating $R_{coll}$, it may be advantageous to use $R_1(t)$ or $R_2(t)$
because Equation \ref{eq: rcoll} typically leads to significant numerical instabilities for high Lorentz factors.

\subsection{Typical Values}

Let us assume shell 1 and 2 have Lorentz factors $\Gamma_1=100$ and $\Gamma_2=400$ and that $\Delta t_e = 0.002$ seconds. The time and radius of collision would be

\begin{align}
	L &= 5.9 * 10^7 \text{ cm} \\ 
	t_{coll} &\approx 42.6 \text{ seconds}\\
	R_{coll} &= 1.2*10^{12} \text{ cm}
\end{align}


\section[Derivations of Gamma_r and Gamma_int]{Derivations of $\Gamma_r$ and $\Gamma_{int}$}

\subsection{Two Shells of Similar Mass}

In a collision between two relativistic shells with Lorentz factors $\Gamma_1$ and $\Gamma_2$, and masses M$_1$ and M$_2$, we can calculate the resulting Lorentz factor of the bulk motion of the merged shells, $\Gamma_r$ and the internal motion of the particles, $\Gamma_{int}$. We can find these quantities by adding the four-momentum of each shell. Since the shells are moving radially in the jet, we only need to consider one axis of propagation in the four vector. If we consider the frame of the resulting merged shell we can write the system as

\begin{align}
	\begin{pmatrix}
	\beta_1\Gamma_1 M_1 c^2\\
	\Gamma_1 M_1 c^2
	\end{pmatrix}
	+
	\begin{pmatrix}
	\beta_2\Gamma_2 M_2 c^2 \\
	\Gamma_2 M_2 c^2
	\end{pmatrix}
	=
	\begin{pmatrix}
	\Gamma_r & \beta_r\Gamma_r\\
	\beta_r\Gamma_r & \Gamma_r
	\end{pmatrix}
	\begin{pmatrix}
	0 \\
	\Gamma_{int}(M_1 + M_2) c^2
	\end{pmatrix}
\end{align}

This system of equations leads to the relations

\begin{align}
	\Gamma_r \Gamma_{int} &= \frac{\Gamma_1M_1 + \Gamma_2M_2}{M_1+M_2} \label{eq: gamrgami 1}\\
	\Gamma_r\beta_r\Gamma_{int}(M_1+M_2) &= \Gamma_1\beta_1M_1+\Gamma_2\beta_2M_2 \label{eq: gamrgami 2}
\end{align}

Using Eq. \ref{eq: gamrgami 2} and substituting in Eq. \ref{eq: gamrgami 1} we obtian

\begin{align}
	\Gamma_r\beta_r\Gamma_{int}(M_1+M_2) &= \Gamma_1\beta_1M_1+\Gamma_2\beta_2M_2 \\
	\beta_r(M_1+M_2) \frac{\Gamma_1M_1 + \Gamma_2M_2}{M_1+M_2} &= \Gamma_1\beta_1M_1+\Gamma_2\beta_2M_2 \\
	\beta_r (\Gamma_1M_1 + \Gamma_2M_2) &= \Gamma_1\beta_1M_1+\Gamma_2\beta_2M_2 \\
	\beta_r &= \frac{\Gamma_1\beta_1M_1+\Gamma_2\beta_2M_2}{\Gamma_1M_1 + \Gamma_2M_2}
\end{align}

By making the assumption that $\Gamma_1, \Gamma_2 >> 1$, we can approximate $\beta \approx 1-\frac{1}{2\Gamma^2}$

\begin{align}
	\beta_r \approx 1-\frac{1}{2\Gamma_r^2} &= \frac{\Gamma_1\beta_1M_1+\Gamma_2\beta_2M_2}{\Gamma_1M_1 + \Gamma_2M_2} \\
	\frac{1}{2\Gamma_r^2} &= 1 - \frac{\Gamma_1\beta_1M_1+\Gamma_2\beta_2M_2}{\Gamma_1M_1 + \Gamma_2M_2} \\ 
	&= \frac{\Gamma_1M_1 + \Gamma_2M_2 - \Gamma_1\beta_1M_1 - \Gamma_2\beta_2M_2 }{\Gamma_1M_1 + \Gamma_2M_2} \\
	&= \frac{\Gamma_1M_1 + \Gamma_2M_2 - \Gamma_1(1-\frac{1}{2\Gamma_1^2})M_1 - \Gamma_2(1-\frac{1}{2\Gamma_2^2})M_2 }{\Gamma_1M_1 + \Gamma_2M_2} \\
	\frac{1}{2\Gamma_r^2} &= \frac{\frac{M_1}{2\Gamma_1} - \frac{M_2}{2\Gamma_2}}{\Gamma_1M_1 + \Gamma_2M_2} \\
	2\Gamma_r^2 &= \frac{\Gamma_1M_1 + \Gamma_2M_2}{\frac{M_1}{2\Gamma_1} - \frac{M_2}{2\Gamma_2}} \\
	&= \frac{\Gamma_1M_1 + \Gamma_2M_2}{\frac{1}{2\Gamma_1\Gamma_2}(\Gamma_2M_1 - \Gamma_1M_2)} \\
	\Gamma_r &= \sqrt{\Gamma_1\Gamma_2\frac{\Gamma_1M_1 + \Gamma_2M_2}{\Gamma_2M_1 - \Gamma_1M_2}} \\
\end{align}

We can assume that shock events occur when two shells of approximately equal mass collide, $M_1 \approx M_2 = M$. This simplifies the above expression to

\begin{align}
	\Gamma_r \approx \sqrt{\Gamma_1\Gamma_2}
\end{align}

To find $\Gamma_{int}$ we return to Eq. \ref{eq: gamrgami 1} and continue to use the assumption $M_1 \approx M_2 = M$, 

\begin{align}
	\Gamma_{int} &= \frac{\Gamma_1M_1 + \Gamma_2M_2}{\Gamma_r(M_1+M_2)}\\
	\Gamma_{int} &= \frac{M(\Gamma_1 + \Gamma_2)}{2M\Gamma_r}\\
	\Gamma_{int} &= \frac{1}{2}\left[ \left(\frac{\Gamma_1}{\Gamma_2}\right)^{1/2} + \left(\frac{\Gamma_2}{\Gamma_1}\right)^{1/2}\right]\\
\end{align}

\subsection{External Shock Considerations}

In the situation of the external shock, we must consider three shells; the reverse shock, the forward shock, and the swept up medium. The reverse and forward shock both have the same Lorentz factor $\Gamma_s$, but they have their own respective mass $M_{RS}$ and $M_{FS}$ and their own respective internal Lorentz factor $\Gamma_{RS,int} = 1$ and $\Gamma_{FS,int} = \Gamma_{int}$. We can discretize the mass of the external medium into chunks of mass $m_{ex}$. Since the external medium is at rest and cold $\Gamma_{ext} = \Gamma_{ext,int} = 1$.

\begin{align}
	\begin{pmatrix}
	\beta_s\Gamma_s M_{RS} c^2\\
	\Gamma_s M_{RS} c^2
	\end{pmatrix}
	+
	\begin{pmatrix}
	\beta_s\Gamma_{int}\Gamma_s M_{FS} c^2 \\
	\Gamma_s M_{FS} c^2
	\end{pmatrix}
	+
	\begin{pmatrix}
	0 \\
	m_{ex} c^2
	\end{pmatrix}
	=
	\begin{pmatrix}
	\Gamma_r & \beta_r\Gamma_r\\
	\beta_r\Gamma_r & \Gamma_r
	\end{pmatrix}
	\begin{pmatrix}
	0 \\
	[M_{RS} + \Gamma_{int}(M_{FS} + m_{ex}) ]c^2
	\end{pmatrix}
\end{align}

This system of equations leads to the relations

\begin{align}
	\beta_s\Gamma_sM_{RS} + \beta_s\Gamma_{int}\Gamma_sM_{FS} &= \beta_r\Gamma_r[M_{RS} + \Gamma_{int}(M_{FS}+m_{ex})] \label{eq: gamma_r_ext}\\
	\Gamma_sM_{RS} + \Gamma_{int}\Gamma_sM_{FS} + m_{ex} &= \Gamma_r (M_{RS} + \Gamma_{int}(M_{FS}+m_{ex})) \label{eq: gamma_int_ext}
\end{align}

We can find $\Gamma_{int}$ directly from Eq. \ref{eq: gamma_int_ext},

\begin{align}
	\Gamma_{int} = \frac{(M_{RS} + \Gamma_{int}M_{FS})\Gamma_s + m_{ex} - \Gamma_rM_{RS}}{\Gamma_r(M_{FS}+m_{ext})}
\end{align}

And using Eq. \ref{eq: gamma_r_ext} we can find $\Gamma_r$,

\begin{align}
	\beta_s\Gamma_sM_{RS} + \beta_s\Gamma_{int}\Gamma_sM_{FS} &= \beta_r\Gamma_r[M_{RS} + \Gamma_{int}(M_{FS} + m_{ex})]\\
	&= \beta_r (\Gamma_sM_{RS} + \Gamma_{int}\Gamma_sM_{FS} + m_{ex})\\
	&= \left(1 - \frac{1}{2\Gamma_r^2}\right) (\Gamma_sM_{RS} + \Gamma_{int}\Gamma_sM_{FS} + m_{ex})\\
	\frac{1}{2\Gamma_r^2} &= 1 - \frac{\beta_s\Gamma_s(M_{RS}+\Gamma_{int}M_{FS})}{\Gamma_s(M_{RS}+\Gamma_{int}M_{FS}) + m_{ex}}\\
	&= \frac{\Gamma_s(M_{RS}+\Gamma_{int}M_{FS}) + m_{ex} - \beta_s\Gamma_s(M_{RS}+\Gamma_{int}M_{FS})}{\Gamma_s(M_{RS}+\Gamma_{int}M_{FS})+m_{ex}}\\
	2\Gamma_r^2 &= \frac{\Gamma_s(M_{FS}+\Gamma_{int}M_{FS})+m_{ex}}{\Gamma_s(M_{RS}+\Gamma_{int}M_{FS})(1-\beta_s) + m_{ex}}\\
	&\text{ use } (1-\beta) \approx \frac{1}{2\Gamma^2}\\
	2\Gamma_r^2 &= \frac{\Gamma_s(M_{FS}+\Gamma_{int}M_{FS})+m_{ex}}{\frac{1}{2\Gamma_s}(M_{RS}+\Gamma_{int}M_{FS}) + m_{ex}}\\
	2\Gamma_r^2 &= 2\Gamma_s\frac{\Gamma_s(M_{FS}+\Gamma_{int}M_{FS})+m_{ex}}{(M_{RS}+\Gamma_{int}M_{FS}) + 2\Gamma_sm_{ex}}\\
	\Gamma_r &= \left[ \frac{\Gamma_s^2(M_{RS} + \Gamma_{int}M_{FS}) + \Gamma_sm_{ex}}{(M_{RS} + \Gamma_{int}M_{FS})+2\Gamma_sm_{ex}}\right]^{1/2}
\end{align}



\section[Derivation of the Minimum Electron Lorentz Factor, Gamma_e]{Derivation of the Minimum Electron Lorentz Factor, $\Gamma_e$} \label{sec: gamma_e}

For each shock collision we can write the mass density $\rho$ and energy density $\epsilon$ (or specific energy density $\epsilon_* = \epsilon/m_p$) within the shock region which let us define the energy densities of the magnetic field $\mu_B$ and the electrons $\mu_e$,

\begin{align}
	\mu_B &= \epsilon_B \rho \epsilon_*\\
	\mu_e &= \epsilon_e \rho \epsilon_*
\end{align}

The number density of accelerated electrons can be written as,

\begin{align}
	n_e^{acc} = \zeta \frac{\rho}{m_p}
\end{align}

where $\zeta$ is the fraction of electrons accelerated in the shock. For the mildly-relativistic forward shock, $\zeta\sim1$, but in order for the relativistic internal shocks to emit MeV $\gamma$ rays, $\zeta\sim10^{-3}$. 

We assume that the electron population is distributed as a power law with index $p$ above a minimum Lorentz factor $\gamma_m$,

\begin{align}
	n(\gamma) = A\gamma^{-p} \text{ for } \gamma\geq\gamma_m
\end{align}

where A is a normalization. We can explicitly find the number density and energy density of the electron population, 

\begin{align}
	n_e^{acc} &= \int_{\gamma_m}^{\infty} n(\gamma)d\gamma = \frac{A}{p-1}\gamma_m^{1-p}\\
	\mu_e &= \int_{\gamma_m}^{\infty} n(\gamma) \gamma m_e c^2 d\gamma = \frac{A}{p-2}\gamma_m^{2-p} m_e c^2
\end{align}

Using the two expressions for $n_e^{acc}$, we can find the normalization A,

\begin{align}
	\zeta \frac{\rho}{m_p} &= \frac{A}{p-1}\gamma_m^{1-p} \\
	&\Rightarrow A = \zeta (p-1) \frac{\rho}{m_p}\gamma_m^{-(1-p)}
\end{align}

We can now rewrite $n(\gamma)$,

\begin{align}
	n(\gamma) &= \zeta (p-1) \frac{\rho}{m_p} \gamma_m^{-(1-p)} \gamma^{-p}\\
	&= \frac{(p-1)}{\gamma_m} \left(\frac{\gamma}{\gamma_m}\right)^{-p} n_e^{acc}
\end{align}

and $\mu_e$

\begin{align}
	\mu_e &= \zeta\frac{\rho}{m_p} \gamma_m^{-(1-p)}\frac{\gamma_m^{2-p}}{(p-2)} m_ec^2\\
	&= \frac{(p-1)}{(p-2)}\gamma_m m_e c^2 n_e^{acc}
\end{align}

which leads us to an expression for $\gamma_m$,

\begin{align}
	\gamma_m &= \frac{(p-2)}{(p-1)}\frac{\mu_e}{n_e^{acc}}\frac{m_e}{c^2}\\
	&\text{where } \frac{\mu_e}{n_e^{acc}} = \frac{(\epsilon_e \rho \epsilon_*)}{(\zeta \frac{\rho}{m_p})} = \frac{\epsilon_e \epsilon_* m_p}{\zeta}\\
	\gamma_m &= \frac{(p-2)}{(p-1)} \frac{\epsilon_e}{\zeta} \frac{m_p}{m_e} \frac{\epsilon_*}{c^2} \\
	\gamma_m &= \frac{(p-2)}{(p-1)} \frac{\epsilon_e}{\zeta} \frac{m_p}{m_e} (\Gamma_{int} -1)
\end{align}

where we used the relation $\epsilon_*/c^2 = (\Gamma_{int} - 1)$ in the final line.

This can be considered in various ways depending on which microphysical parameters are kept constant or allowed to vary. For example, if we let $\zeta \sim \epsilon_* c^2$ then $\gamma_m$ will remain constant.

\section[Derivation of nu_m and nu_c]{Derivation of $\nu_m$ and $\nu_c$}

We recall that the frequency associated with a particular Lorentz factor is defined as

\begin{align}
	\nu(\gamma) = \gamma^2 \frac{q_e B}{2\pi m_e c}
\end{align}

Following the prescription of $\gamma_m$ (see Sec. \ref{sec: gamma_e}) we can find the frequency associated with the minimum Lorentz factor of an electron distribution (see \textit{Radiative Processes} Chapter 6.1 of Rybicki and Lightman, 1979),

\begin{align}
	\nu_m &= \nu(\gamma_m) = \gamma_m^2 \frac{q_e B}{2 \pi m_e c}\\
	&= \frac{(p-2)^2}{(p-1)^2} \frac{\epsilon_e^2}{\zeta^2} \frac{m_p^2}{m_e^2} \frac{\epsilon_*^2}{c^4} \frac{q_e B}{2 \pi m_e c}\\
	\nu_m &= \frac{1}{2\pi} \left(\frac{(p-2)}{(p-1)} \frac{\epsilon_e}{\zeta}\right)^2 \frac{m_p^2 q_e}{m_e^3 c^5} \epsilon_*^2 B
\end{align}

To find the critical Synchrotron frequency $\nu_c$, we musts first find the associated critical Synchrotron Lorentz factor $\gamma_c$. We do this by setting the energy of an electron accelerating in a magnetic field equal to the power generated in a time $t$ by the same electron,

\begin{align}
	\gamma_c m_e c^2 &= P(\gamma_c) t\\
	&= \frac{4}{3} \sigma_T	c \gamma_c^2 \frac{B^2}{8\pi} t\\
	\gamma_c &= m_e c^2 \frac{3}{4} \frac{1}{\sigma_T c} \frac{8\pi}{B^2t}\\
	\gamma_c &= \frac{6 \pi m_e c}{\sigma_T} \frac{1}{B^2t}
\end{align}

which enables us to find $\nu_c$,

\begin{align}
	\nu_c &= \nu(\gamma_c) = \left(\frac{6\pi m_e c}{\sigma_t} \frac{1}{B^2t}\right)^2 \frac{q_e B}{2\pi m_e c}\\
	&= \frac{36\pi^2 m_e c^2}{\sigma_T^2} \frac{q_e}{2\pi m_e c}\frac{1}{B^3t^2}\\
	\nu_c &= \frac{18\pi q_e m_e c}{\sigma_T^2} \frac{1}{B^3t^2}
\end{align}


\section{Simulation Time Considerations}

In this section we make rough estimates of the times required to run all simulations required for a particular parameter space. Each single simulation requires $\sim$ 5 seconds to run.

\subsection{Using observations to help constrain the parameter space}

Using the non-thermal component observed GRB prompt emission spectra, we can help limit the parameter space we must explore during a simulation. 

\begin{align}
	& [t_w] &&\times [\Gamma(t)] &&\times [\zeta]  \\
	& [3] &&\times [3\times3\times4] &&\times [3] = 432 \text{ simulations}
	& 432 \text{ simulations} \times 5 \text{ sec} \sim 36 \text{minutes}
\end{align}

this will need to be done for every time bin used, because $\dot{E}^k_{iso}$ must be changed depending on the observed brightness.

If the thermal component is not observed, this part of the simulation can be removed. If the thermal component is observed, the brightness can be used to constrain the value of $\epsilon*(1+\sigma)\leq1$.

\subsection{Creating a library}

To create a library that can be applied to any GRB prompt emission, the parameter space will be much more broad. 

\begin{align}
	[t_w] \times &[\sigma] \times &&[\dot{E}] \times&&[\Gamma(t)] \times&&[\zeta] \times&&[\epsilon_{th}] \times&&[\epsilon_{e}] \times&&[\epsilon_{B}] \\ 
	[5] \times &[3] \times &&[5] \times&&[3\times3\times4] \times&&[4] \times&&[4] \times&&[2] \times&&[4] = 345,600 \text{ simulations} \\
\end{align}

\begin{align}
	345,600 \text{ simulations} \times 5 \text{ sec} \sim 20 \text{ days}
\end{align}

This is obviously not feasible.

\section{Custom Classes/Libraries}

Below is a list of the classes and libraries created for the GRB prompt emission simulation code.

\begin{deluxetable*}{ll}
\tablecaption{ Custom classes and libraries created for the GRB prompt emission simulation code. \label{tab: custom packages}}
\tablewidth{0pt}
\tablehead{
\colhead{Library Name} & \colhead{Description}
}
\startdata
cosmology.cpp & Defines useful cosmology constants and functions used throughout the code. \\
utilfuncs.cpp & Defines useful utility functions used throughout the code. \\
Spectrum.cpp & Defines all necessary variables and functions that pertain to a spectrum object. \\
LightCurve.cpp & ``'' light curve object. \\
TTEs.cpp & ``'' Time Tagged Event object. \\
Response.cpp & ``'' Response object. \\
FitStats.cpp & Calculates and stores fit statistics values during fitting. Includes read/write methods. \\
ShellDist.cpp & Defines the functions used to calculate the Lorentz for each shell in the wind. \\
ModelParams.cpp & Stores the model parameters currently being used. Includes read/write methods. \\
SynthGRB.cpp & Contains the necessary scripts to simulate GRB prompt emission. Calculates the spectra and light curves. \\
SynthGRBLibrary.cpp & Creates multiple SynthGRB's based on user defined parameter space to create a library. \\
ObsGRB.cpp & Contains methods to store and manipulate observed GRB data. \\
DataAnalysis.cpp & Contains all methods used for data manipulation and fitting. \\
\enddata
\end{deluxetable*}

\section{Packages Used}

In Table \ref{tab: standard packages} I list the standard C++ and Python packages used.

\begin{deluxetable*}{ll}
\tablecaption{A list of the standard C++ packages used in the simulation code and Python packages used for plotting. \label{tab: standard packages}}
\tablewidth{0pt}
\tablehead{
\colhead{C++ Package Name} & \colhead{Python Package Name}
}
\startdata
cmath & matplotlib.pyplot \\
cstdio & numpy \\
cstring & os \\
iostream & scipy.integrate (to calculate luminosity distance) \\
fstream & \\
vector & \\
sstream & \\
\enddata
\end{deluxetable*}


\end{appendix}

\end{document}
