\documentclass[linenumbers,twocolumn]{aastex631}
% \documentclass[linenumbers]{aastex631}

% Packages
\usepackage[utf8]{inputenc}
\usepackage{graphicx}
\usepackage{amsmath}
\usepackage{amssymb}
\usepackage{enumitem}
\usepackage{ulem}
\usepackage{hyperref}

% Editing commands
\newcommand{\mm}[1]{{\textcolor{purple}{\bf #1}}}

% Make upright subscripts and superscripts in Mathmode.
\def\subinrm#1{\sb{\mathrm{#1}}}
{\catcode`\_=13 \global\let_=\subinrm}
\mathcode`_="8000
\def\supinrm#1{\sp{\mathrm{#1}}}
{\catcode`\^=13 \global\let^=\supinrm}
\mathcode`^="8000
\def\upsubscripts{\catcode`\_=12 } \def\normalsubscripts{\catcode`\_=8 }
\def\upsupscripts{\catcode`\^=12 } \def\normalsupscripts{\catcode`\^=7 }

\newcommand{\vdag}{(v)^\dagger}
\newcommand\aastex{AAS\TeX}
\newcommand\latex{La\TeX}

% Title
\shorttitle{Deriving the Shock Crossing Time}
\shortauthors{Moss M.} 

\begin{document}

\upsubscripts
\upsupscripts

\title{Deriving the Shell Crossing Time as a Function of the Density}

\correspondingauthor{Michael Moss}
\email{mikejmoss3@gmail.com}

\begin{abstract}
The goal of this derivation to estimate the time it will take a reverse shock to cross incoming ejecta material.

\end{abstract}

\section{Introduction}
{
    In the refreshed shock model, the wind of a GRB can be separated into two regions, (i) early outflow launched with $\bar{\Gamma}\sim150$ and (ii) later ejecta launched launched with $\bar{\Gamma}\sim15$. The early material is responsible for producing the prompt emission and the afterglow continuum emission. The later ejecta will eventually catch up to the early ejecta as it sweeps up circumburst medium and decelerates. When the later ejecta catches up and collides with the early material, a ``refreshed'' shock occurs, injecting energy into the front of the outflow and may be witnessed as bumps in the afterglow light curves (potentially in the optical regime). In this work, we would like to estimate the time it takes the incoming energy injected from the later ejecta to be distributed across the newly shocked material. This timescale should be dominated by the shell crossing time $t_{\Delta }$, where we model the incoming late ejecta as a shell with width $\Delta$, density $n_4$, and Lorentz factor $\gamma_{4}\sim15$. 

    To estimate $t_{\Delta}$, we must estimate the density of the rapid material ($n_4$) and the material which the late ejecta is colliding with ($n_1$).
}

\section{Estimating Density of Zone 1 Material}
{
    The material that the late ejecta is colliding with is composed of early ejecta material that has been crossed by the reverse shock and has time had time to relax. The average particle density of this material $n_1$ (fluid frame) can be approximated as,

    \begin{align}
        n_{ej}(t) &= n_{1}(t) = \frac{E_{iso}\Gamma(t)}{4\pi m_p c^2 \Gamma_0 R(t)^2 \Delta_1(t)} \label{eq: dens estim} \\
    \end{align}

    where $R(t)$ is the distance of the material from the central engine, $E_{iso}$ is the isotropic equivalent energy, $\Gamma_0$ is the initial bulk Lorentz factor of the ejecta, and $\Gamma(t)$ is the bulk Lorentz factor as a function of time (all in the central engine frame). $\Delta_1(t)$ is the initial width of the material. 

    $\Delta_1(t)$ can take two extremes, if we assume that the early material relaxes as it propagates along the jet then $\Delta_1(t) \approx R(t)/\Gamma_1^2(t)$, and the density approximations becomes

    \begin{align}
        n_{1,min}(t) = \frac{E_{iso}\Gamma(t)}{4\pi m_p c^2 \Gamma_0 R(t)^3} \label{eq: dens estim min} \\
    \end{align}

    In the other extreme, if the material does not relax and the width of the material is approximately $\Delta \sim \beta_0 ct_{inj}$ where $t_{inj}$ is the duration of the injection time and $\beta_0 = \sqrt{1 - 1/\Gamma_0^2}$. This makes the approximation of the density 

    \begin{align}
        n_{1,max}(t) = \frac{E_{iso}\Gamma(t)}{4\pi m_p c^3 \Gamma_0 R(t)^2 t_{inj}} \label{eq: dens estim max} \\
    \end{align}

    We can estimate this density using a fiducial values $E_{iso} = 10^{53}$ erg/s, $\Gamma_0 = 150$, $R(t_{coll}) = 10^{17}$ cm, $\Gamma (t_{coll}) = 5$, and $t_{inj} = 20$ sec, which leads to a density estimate of $n_{1,min}=1.8\times10^{2}$ cm$^{-3}$ and $n_{1,max} = 3 \times 10^{7}$ cm$^{-3}$.
}

\section{Estimating the Shell Crossing Time}
{
    The width of the late time ejecta can be described as

    \begin{align}
        \Delta(t) = \Delta_0 + R(t)/\gamma^2(t)
    \end{align}

    where $\Delta_0 = \beta c t_{inj}$ is the approximate width of the material launched from the central engine over an interval $t_{inj}$ and $R(t)/\gamma^2(t)$ accounts for the spread of the material as it propagates away from the central engine. The minimum width assumes there is no spreading, i.e., $\Delta_{min} \approx \Delta_0$. The maximum width will be dominated by the spread $\Delta_{max}\approx R(t)/\gamma^2(t)$. For the relevant parameter values of our scenario

    \begin{align}
        \Delta_{min} &\approx 6\times10^{11} \text{ cm}\\
        \Delta_{max} &\approx 4.5\times10^{14} \text{ cm}
    \end{align}

    Following Sari and Piran 1995 (hereafter SP95), the observed timescale of emission is

    \begin{align}
        \Delta t_{obs} &= R_e / \gamma^2_4c \label{eq: t_obs} \\
        &= \bigg\{ \begin{array}{ll}
            \Delta/c, & \text{if } \xi < 1\text{ (Relativistic)}\\
            l\gamma^{8/3}c, & \text{if } \xi > 1 \text{ (Newtownian)}
        \end{array}
    \end{align}

    where $l = (E/n_1 m_p c^2)$ is the Sedov length and we have replaced their $\gamma_2$ with $\gamma_4$ as the relevant Lorentz factor for our incoming thin-shell situation. According to the density ratio we are within the relativistic regime, i.e., $f = n_4/n_1 << \gamma^2$, but when estimating the dimensionless quantity $\xi = (l/\Delta)^{1/2} \gamma^{-4/3}$ we are in the commonly in the Newtownian regime, i.e., $\xi > 1$. Luckily, we do no have to worry about this, because estimating $\Delta t_{obs}$ in the relativistic regime using $\Delta_{min}$ and $\Delta_{max}$ encompasses the Newtownian estimate.

    For our situation, we would like $\Delta t_{obs} = \Delta/c \lesssim 3\times10^4$ sec, which gives the upper limit to the shell width $\Delta \lesssim 9\times10^{14}$ cm, which is greater than $\Delta_{max}$. So, I think we are good. 
}

\section{Estimating Spread of The Lorentz Factor}
{
    We can use the spread of the late time material as an upper limit to what the possible spread of the Lorentz factor can be in the late material. Consider two photon launched at the same time and position with Lorentz factor $\gamma$ and $\gamma+\delta \gamma$. At a later time $t$, the distance between the two photons can be expressed as

    \begin{align}
        \Delta &= \beta_f ct - \beta_s ct \\
        &= ct\left[\left(1-\frac{1}{2(\gamma+\delta \gamma)^2}\right)- \left(1 - \frac{1}{2\gamma^2}\right)\right]\\
        &\approx R \frac{\delta \gamma}{\gamma^3} \label{eq: delta}
    \end{align}

    where $\beta_f$ and $\beta_s$ are the defined for the faster and slower photons, respectively. The dynamical timescale of the system is determined by the Lorentz factor of the early material. In principle, the early material is constantly decelerating as it sweeps up circumburst material, but to an order unity we can write 

    \begin{align}
        t_{dyn} &\approx \frac{R}{2c\gamma_1^2} \\
        &\Rightarrow \frac{R}{c} \approx 2 \gamma_1^2 t \label{eq: t dyn}
    \end{align}

    Substituting Equation \ref{eq: delta} and \ref{eq: t dyn} into Equation \ref{eq: t_obs} (assuming we are in the relativistic regime), we obtain

    \begin{align}
        \Delta t_{obs} &= \frac{\Delta}{c} \\
        &= \frac{R}{c} \frac{\Delta \gamma}{\gamma^3}\\
        &= 2 t \frac{\Delta \gamma}{\gamma} \left(\frac{\gamma_1}{\gamma}\right)^2 \\
        \frac{\Delta t}{t} &= 2 \frac{\Delta \gamma}{\gamma} \left(\frac{\gamma_1}{\gamma}\right)^2
    \end{align}

    Using the observed timescale from the rise times of the jumps in the light of GRB 030329, i.e., $\Delta t_{obs}/t_{obs} \sim 0.3$, and assuming that the early and late ejecta have a Lorentz factor ratio of $\gamma_1/\gamma_4 \sim 1/3$ (or $\sim 1/2$), we can constrain the spread of the Lorentz factors as a function of the average Lorentz factor, i.e., $\delta \gamma / \gamma  \sim 1.14$ (or $\sim 0.6$).
}

    
\end{document}
